\chapter{Recommendations}
Another way would be to refactor the simulation enough so that they can be ran on GPU using CUDA. The advantage would be that we would be able to speed up the simulation significantly with our existing setup. The difficulty is that due to single core optimization implemented by some of the classes, it is required for every thread to have its own copy of the entire setup. In the CUDA environment where on--board memory comes at a premium, this would require some major code refactoring and testing before any development can take place, and it may not be as cost--efficient as getting a new CPU which may also provide significant speed improvement. Having multiple GPUs enabled also proved to destabilize the Windows 10 system on our current setup, causing the system to crash by DPC\_WATCHDOG\_VIOLATION.  Further, the code is written in a completely CUDA--compliant manner, and a blueprint of the working principles is already in place. Barring any hardware upgrade or new insights in working details of CUDA API, the development of a working CUDA implementation of the simulator would be indefinitely postponed.