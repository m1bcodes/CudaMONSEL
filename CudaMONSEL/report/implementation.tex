\chapter{Implementation}\label{impl}
In this section we discuss the implementation details of CudaMONSEL.

The simulation is divided into two main packages, microanalysis and nanoanalysis, based on their respective functionalities and thus, authors. The microanalysis package, authored by Nicholas Ritchie, deals primarily with x--ray analysis, while the nanoanalysis package, authored by John Villarrubia, focuses on electron's interactions with materials. While their applications are different, nanoanalysis depends on microanalysis' implementation (mostly geometry) to work. We will introduce both sections below, with an emphasis on the connections between the two packages.

\section{Microanalysis}
In this package, the physics of the simulation are provided by subclasses of \emph{AlgorithmUser} and \emph{AlgorithmClass} abstract classes.

\subsection{Algorithm Classes}
This section is dedicated to the implementation details of classes that are related the physics of the simulation. All of these classes are derived from the abstract base class algorithm 

The main purpose of \emph{AlgorithmUser} class is to provide the abstract method \emph{initializeDefaultStrategy} which provides a function shared by its subclasses to set each class in a proper state required to run the physics. 

The secondary purpose of this class is to provide a place for the simulation to define the default algorithms to use when none is specified. This is enabled by the \emph{Strategy} class.

The \emph{AlgorithmClass} abstract class inherits from \emph{AlgorithmUser} abstract class and included more information about the source of the physics algorithm (eg author). It also provides a way to query all physical algorithms of the same type (TODO: explain CLEARLY) via the \emph{getAllImplementations} abstract method.

\subsubsection{}

\section{Simulation}\label{impl:sim}
The simulation computes the movement of the electron for each unit of time step. First, it acquires the active electron's position. From the position info, it calculates the current active region and the associated material scattering model. To obtain the position of electron at the next time step, the simulation acquires a random mean path length of the electron based on the scattering model and computes the electron's position at next time step. With the new position, the simulation obtains the next region, moves the electron and decreases its energy. Finally, it perfoems a check to determine if the active electron's trajectory is completed. If the trajectory is not completed, the simulation performs one of three actions based on the next region. At each time step, the energy loss and position of the electron is updated. When a new electron is spawned, the original electron is stored on a stack until the new electron is destroyed when its energy falls below the tracking threshold of the scattering mechanism. Refer to function \emph{takeStep} in MonteCarloSS.cu for details. 

\section{Material}\label{impl:mat}
\subsection{Material}
\subsection{Secondary Eleectron Generating Material}
To define an SE material, one will specify
\begin{enumerate}
\item composition (what elements and in what proportions)
\item density
\item work function (optional)
\item work function (optional)
\item energy of its conduction band minimum (optional)
\item band gap (optional)
\item core energy levels (optional)
\item its name (optional)
\end{enumerate}

The necessity of the optional parameters depends on the model associated with the material, that is, they are only required if they are needed by the model.

\section{Barrier Scattering Mechanism}\label{impl:barrier_sm}
\begin{lstlisting}[caption={BarrierScatterMechanism.cuh},label={lst:barrierSM},numbers=left,escapeinside={@}{@}]
class BarrierScatterMechanism
{
public:
   virtual ElectronT* barrierScatter(ElectronT* pe, const RegionBaseT* nextRegion) const = 0;
};
\end{lstlisting}

The barrier scatter mechanism follows the abstract class shown in \ref{lst:barrierSM}.

\subsection{Exponential Quantum Mechanical Barrier Scatter Mechanism}
A barrier scattering mechanism is associated with each region of our sample, but barrier transmission is a pair-wise phenomenon. That is, the barrier height and width at an interface between materials A and B depends in principle upon the properties of both materials. If the electron starts inside material A, then it is material A's barrier scatter mechanism that governs that particular scattering event. The present scatter mechanism determines the barrier height by comparing the potential energies in the materials on each side of the interface. It uses the barrier width associated with material A. This scattering mechanism uses the property energyCBbottom, which therefore need to be properly defined for the materials on both sides of the interface.

This class computes barrier transmission using the formula provided by \cite{Landau}. Note that this is the only barrier SM available in the JMONSEL package. 

\section{Scatter Mechanism}\label{impl:sm}
The prototype of the class is shown below
\begin{lstlisting}
class ScatterMechanism
{
public:
   /**
    * Returns the reciprocal of the mean free path.
    *
    * @param pe - the primary electron
    * @return double Reciprocal of mfp in inverse meters
    */
   virtual double scatterRate(Electron pe) = 0;

   /**
    * Updates properties of the primary electron based on results of scattering and returns either a secondary electron or null.
    * @param pe -- the primary electron
    * @return Electron -- the generated secondary electron or null
    */
   virtual Electron scatter(Electron pe) = 0;

   /**
    * Sets the material within which the electron scatters. This method typically precomputes and caches combinations of material properties required by the scattering model.
    * @param mat
    */
   virtual void setMaterial(Material mat) = 0;
}
\end{lstlisting}

Scatter mechanisms included in JMONSEL are:
\begin{enumerate}
\item BrowningMottElasticSM
\item FittedInelSM
\item GanachaudMokraniPhononInelasticSM
\item GanachaudMokraniPolaronTrapSM
\item KoteraPlasmonInelasticSM
\item MollerInelasticSM
\item SelectableElasticSM
\item TabulatedInelasticSM
\end{enumerate}

A model usually requires two scattering mechanism, one for elastic scattering and one for inelastic / SE--generating scattering. For elastic scattering, SelectableElasticSM with NISTMottRS is a good choice. (cite) For inelastic scattering, choose TabulatedInelasticSM pair with ZeroCS) when when scattering tables are available, and use FittedInelasticSM (pair with JoyLuoNieminenCSD) otherwise. For most models, the governing equations are hardcoded into the class itself. 

\subsection{FittedInelSM}
FittedInelSM requires a parameter, $E_{av}$, that represents the average energy required to make a secondary electron. It is based on an old semi--empirical model, described most recently (and perhaps in most refined form) by \cite{lin2005new}.

\subsection{TabulatedInelasticSM}
For TabulatedInelasticSM, most of the scattering properties are not hardcoded into the object itself, but read from tables associated with each supported material. For example, the inverse mean free path, interpolating an IIMFP table associated with the material, the distribution of energy losses and angle changes by the primary electron are similarly determined by tables. 

The tables are served as strings of full paths to the files containing the physical data. Each string is input to NUTableInterpolation object. The NUTableInterpolation object expects the text files to be in the following format: first entry in the file should be number of dimensions of domain $n$. The next entry is the number of entry in the first dimension $n_1$, then followed by $n_1$ entries representing values in the first dimension, followed by the number $n_2$ etc. The numbers form a $n$ dimensional mesh that represents the domain of interpolation. After repeating the pattern for $n$ times, the file should contain $\prod_{i=1}^{n}n_i$ numbers representing the values of the data points. 

The tables are:
\begin{enumerate}
    \item tableIIMFP, IIMFP 1--d table (inverse inelastic mean free path vs. primary electron energy $E_O$)
    \item tableReducedDeltaE, reduced deltaE 2--d table ($\delta E/E_0$ vs $E_0$ and $r$, with $r$ a random number)
    \item tableTheta, the theta 3--d table (scattering angle of PE vs. $E_0$, $\delta E/(E_0-E_{Fermi})$, $r$)
    \item tableSEE0, the 2--d table of SE initial energy vs. $\delta E$ and $r$
\end{enumerate}

For 3 of the tables, the first input parameter is the kinetic energy of the primary electron. By default, this kinetic energy is assumed to be the PE's energy with respect to the bottom of the conduction band. Sometimes, however, tables use a different convention. For example, in a semiconductor or insulator tables may be computed for energies measured with respect to the bottom of the valence band (the highest occupied band). This can be accommodated by providing the constructor with the energy offset between these two definitions of energy (ie., $E_{offset} = \text{energy of conduction band bottom} - \text{the energy defined as the zero for purpose of the tables}$).

However, an exception to the above generalizations. There remains some uncertainty in the literature over the connection between primary electron energy loss/trajectory change on the one hand and secondary electron (SE) final energy/final trajectory on the other. Part of the reason for this difference lies in varying assumptions about the initial (pre-scattering) energy and trajectory of the SE. Some of the different methods of treating this connection have been implemented in this class. One of them must be selected to instantiate an object of this class. The selection is made by choosing a value for the methodSE parameter in the constructor. Three methods have been implemented to deal with this.

The tables are found under the directory ``C:\textbackslash Program Files\textbackslash NIST\textbackslash JMONSEL\textbackslash ScatteringTables''.

\begin{enumerate}
    \item $methodSE=1$ \cite{ding1996monte}: If the PE energy loss, $\delta$ E is greater than a core level binding energy, the SE final energy is $\delta E-E_{binding}$. Otherwise, it is $\delta E + E_{Fermi}$, where $E_{Fermi}$ is the Fermienergy of the material. The final direction of the SE is determined from conservation of momentum with the assumption that the SE initial momentum was 0.
    \item $methodSE=2$ \cite{ding2001monte}: If $\delta E$ is greater than a core level binding energy the treatment is the same as $methodSE=1$. If not, the SE final energy is $\delta E + E'$. If $E'$ were the Fermi energy this would be the same as $methodSE = 1$. However, $E'$ lies in the range $max(0, E_{Fermi} - \delta E) <= E' <= E_{Fermi}$. The value of $E'$ is determined probabilistically based upon the free electron densities of occupied and unoccupied states.
    \item $methodSE=3$ \cite{mao2008electron}: The scattering event is assigned as either an electron--electron event or an electron--plasmon event based upon the momentum transfer. Plasmon events are treated as in $methodSE = 2$, except that the SE direction is isotropic. Electron--electron events have energy and direction determined as described by \cite{mao2008electron}.
\end{enumerate}

To use this class, the user--provided tables must cover the full range of energies that the electron will encounter in the simulation. This generally means the tables that take PE energy as one of the inputs should cover energies from the material's Fermi energy up to at least the energy of electrons generated by the electron gun.

This class uses coreEnergy array, energyCBbottom, bandgap, and workfunction.

\section{Randomized Scatter}
The type of randomized scatter depends on the the energy of the electron. The main computations are performed by these two member functions below

\begin{lstlisting}
__host__ __device__ virtual double randomScatteringAngle(const double energy) const = 0;
__host__ __device__ virtual double totalCrossSection(const double energy) const = 0;
\end{lstlisting}

The method \textit{totalCrossSection} computes the total cross section for an electron of the specified energy.

The method \textit{randomScatteringAngle} returns a randomized scattering angle in the range [0,PI] that comes from the distribution of scattering angles for an electron of specified energy on an atom of the element represented by the instance of the class.

There is also a method that tracks the type of element the class is applicable to. Below we introduce each type of randomized scatter in the simulation.

\subsection{CzyzewskiMottCrossSection}
\subsection{BrowningEmpiricalCrossSection}
Browning empirical cross section is applicable to low energy electrons (although not lower than $100eV$). 
\begin{align*}
   CS(Z, e) &= \frac{3\times 10^{-22} Z^{1.7}}{e + 0.005 Z^{1.7} \sqrt{e} + 0.0007 Z^{2} / \sqrt{e}}
\end{align*}
where $e$ is electron energy in keV, $Z$ is the proton number of the element.

\begin{align*}
    SA(Z, e)= 
\begin{cases}
    acos\left (1.0 - \frac{2.0 \alpha r_2}{\alpha - r_2 + 1}\right ),& \text{if } r_1 \leq \frac{r}{r+1}\\
    acos(1-2.0r_2),& \text{otherwise}
\end{cases}
\end{align*}
where $\alpha = 7.0\times 10^{-3}/e$, $r_1, r_2$ are random numbers $\in [0, 1]$, and $r = (300.0 e / Z) + (Z^3 / 3.0\times 10^5 e)$.

\subsection{ScreenedRutherfordScatteringAngle}
Screened Rutherford cross section is applicable to high energy ($\geq 20keV$) electrons. 
\begin{align*}
   CS(Z, e) &= \frac{7.670843088080456\times 10^{-38} zp (1.0 + z)}{e + 5.44967975966321\times 10^{-19} zp^2}
\end{align*}
where $e$ is the electron energy in keV, $Z$ is the proton number, and $zp = Z^{1.0/3.0}$.

\begin{align*}
    SA(Z, e) &= acos\frac{1 - 2.0 * alpha * r}{1 + alpha - r}
\end{align*}
where $\alpha = 5.44968\times 10^{-19} Z^{2.0 / 3.0} / e$, $r \text{is a random numbrs} \in [0, 1]$.

\subsection{NISTMottScatteringAngle}
\subsection{NISTMott}
NISTMott randomized scatter tries to provide a comprehensive model for all energy ranges. It is a function of electron energy and material/element the electron is colliding with. For energy between $100eV$ and $20keV$, it calculates using the the NIST SRD 64 method by interpolating from a table of cross sections with Lagrange interpolation. For energies lower than $100eV$, it uses one of three methods. 
\begin{enumerate}
\item Cutoff energy is $50 eV$, interpolation Browning,
\item Cutoff energy is $100 eV$, interpolation Browning,
\item Cutoff energy is $100 eV$, linear interpolation
\end{enumerate}
In all three methods, the cross section is zero when the energy is zero, and the cross section matches the tabulated value when energy is at some minimum cutoff value, which may or may not be the minimum tabulated value at $50eV$. Findally, for energies above the tabulated range, computations use Screened Rutherford Scattering Angle.

\begin{appendices}
\chapter{Lagrange Interpolation}
Given $n+1$ points $(x_0, y_0)$, ... $(x_n, y_n)$, the Lagrange Interpolation is given by 
\begin{align*}
    f_n(x) = \sum^{n}_{i=0} L_i(x)f(x_i)
\end{align*}
where $f(x_i) = y_i$ and $L_i(x)=\prod_{j=0,j\ne i}^{n}\frac{x-x_j}{x_i-x_j}$.
\end{appendices}

NISTMottScatteringAngle uses 2nd order Lagrange interpolation in $log(E)$ for the totalCrossSection calculation. For randomScatteringAngle it uses $0^{th}$ order interpolation (i.e., it chooses the nearest tabulated value) in $log(E)$ and 1st order in the random number. In contrast the present class uses 3rd order for all interpolations. Orders can be set by the constants \textit{qINTERPOLATIONORDER} and \textit{sigmaINTERPOLATIONORDER}. The interpolation tables are given by
\begin{enumerate}
\item $mSpwem$ is a 1--d table of total cross sections, uniform in $log(E)$ from $log(50.)$ to $log(20000)$
\item $mX1[j][i]$ is a 2--d table of $q$ vs $Log(E)$ and random \#. $j$ indexes $log(E)$. $i$ indexes $r$. $q$ is related to $cos(\theta)$ by $1-2q^2=cos(theta)$
\end{enumerate}

% \begin{align*}
%    CS(Z, e) =
%    \begin{cases}
%       \text{browning SA}, \text{if } e \lt 
%    \end{cases}
% \end{align*}
% where $e$ is electron energy in keV, $Z$ is the proton number of the element.

\begin{align*}
    SA(Z, e)= 
\begin{cases}
    acos\left (1.0 - \frac{2.0 \alpha r_2}{\alpha - r_2 + 1}\right ), & \text{if } r_1 \leq \frac{r}{r+1}\\
    acos(1-2.0r_2), & \text{otherwise}
\end{cases}
\end{align*}
where $\alpha = 7.0\times 10^{-3}/e$, $r_1, r_2 \text{are random numbrs} \in [0, 1]$, and $r = (300.0 e / Z) + (Z^3 / 3.0\times 10^5 e)$.

\section{Slowing Down Algorithm}\label{impl:sda}
\begin{lstlisting}
public interface SlowingDownAlg {
   /**
    * Sets the material for which the energy loss is to be computed
    *
    * @param mat
    */
   void setMaterial(SEmaterial mat);

   /**
    * compute - Computes the energy change for an electron with initial energy
    * eK traversing distance d. The return value is negative if the electron
    * loses energy.
    *
    * @param d double -- the distance moved by the electron
    * @param pe Electron, the primary electron
    * @return double -- the energy change
    */
   double compute(double d, Electron pe);
}
\end{lstlisting}
\subsection{Joy--Lou--Nieminen}
The JoyLuoNieminenCSD stopping power requires a parameter, breakE, that specifies where the stopping power switches from the low energy Nieminen expression to the higher energy Joy/Luo one.

\section{Events}\label{impl:events}
The event listener `hooks' onto the simulation and provides a way to transfer information out of the simulation loop. This means that event listeners are by design passive and unable to affect the outcome of the simulations. The simulation will work even when all event--related code are removed, in which case some other mean of acquiring simulation state needs to be implemented if one wishes to retrieve any information about the on--going simulation.

The following events are defined inside $MonteCarloSS$ class:
\begin{enumerate}
\item ScatterEvent
\item NonScatterEvent
\item BackscatterEvent
\item ExitMaterialEvent
\item TrajectoryStartEvent
\item TrajectoryEndEvent
\item LastTrajectoryEvent
\item FirstTrajectoryEvent
\item StartSecondaryEvent
\item EndSecondaryEvent
\item PostScatterEvent
\item BeamEnergyChanged
\end{enumerate}

\subsection{Detector}\label{subsec:detector}
The most important events of are events relating to detecting backscattered electrons. Detectors are devices that collect the information about electrons for later processing. In the simulation, electrons that are characterized as secondary electrons are backscattered electrons with energy below $50 eV$. To simulate the detector, backscatter events are used. Backscatter events are fired only when the electron hits the chamber wall (one of the three normal end states of a electron). If required, we may also obtain forward--scattered secondary electrons for TEM image simulations.

Concretely, the event listener must be added to the simulation class (ie $MonteCarloSS$), before the start of the simulation, and have member function $actionPerformed$ of the listener classes added to the simulation deal with the appropriate events. Note that, the simulation broadcasts the event triggered to all listener classes by calling the member function $actionPerformed$ of every listener class added to the simulation in the function $fireEvent$.

\section{Amphibian -- custom C++ library}
Since one of the goals of the project is to make MONSEL run on both CPU and GPU, we built a custom library, Amphibian, to eliminate any dependence on the C++ standard template library (STL). The following data structures are included,
\begin{enumerate}
\item linkedlist (single-- and double--linked)
\item vector
\item stack
\item unordered set
\item unordered map
\item (mutable) string
\end{enumerate}

Algorithms like quick sort, binary search are also included. There is also a random class responsible for generating random numbers reliably.
