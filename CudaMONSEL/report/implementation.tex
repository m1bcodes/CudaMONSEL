\chapter{Implementation}
\section{Simulation}
In the simulation, the movement of the electron is computed for each unit of time step. First, the active electron's position is obtained. From the position info, the current active region and the associated material scattering model is obtained. To obtain the position of electron at the next time step, the simulation acquires a random mean path length of the electron based on the scattering model and computes the electron's position at next time step. With the new position, the simulation obtains the next region, moves the electron and decreases its energy. Finally, the simulation perfoems a check to determine if the active electron's trajectory is completed. If the trajectory is not completed, the simulation performs one of three actions based on the next region. Refer to function \emph{takeStep} in MonteCarloSS.cu for details.

\section{Material}\label{sec:impl_mat}
\subsection{Material}
\subsection{Secondary Eleectron Material}
%\subsubsection{Antenna}

\section{Barrier Scattering Mechanism}\label{sec:impl_barrierSM}
\begin{lstlisting}
class BarrierScatterMechanism
{
public:
   virtual ElectronT* barrierScatter(ElectronT* pe, const RegionBaseT* nextRegion) const = 0;
};
\end{lstlisting}
