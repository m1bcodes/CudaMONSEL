\chapter{Introduction}
\pagestyle{fancy}
\fancyhead{}
\fancyhead[R]{\thepage}
\fancyfoot{}
\renewcommand{\headrulewidth}{0pt}
\renewcommand{\footrulewidth}{0pt}

\pagenumbering{arabic}
\setcounter{page}{1}
Driven by Moore's Law, the size of transistors has shrunk to nanometer scale. The small size of the transistors posts significant challenges for semiconductor failure analysis as it becomes increasingly difficult to obtain clear images of the device.

The most common technique to acquire images of semiconductor is scanning electron microscopy (SEM). SEM scans the object being image and collects the low energy secondary electrons, which are electrons produced from the object itself by chain reaction due to the higher energy primary electron. Several problems are presented when using this techniques to acquire images. For example, at high magnification there are significantly more electrons produced at around the edge of object to to increased surface area, which results in the edges being brighter than the rest of the object. This ``edge bloom'' effect affects the measurement of critical dimensions as it is difficult for software to identify the edge during the post procesing step. On the other hand, the boundaries between materials are unclear sometimes. These problems call for a better method for measurement.

In recent years, machine learning has proven to be a powerful end--to--end solution in many area of data science, including natural language processing and image analysis. In particular, convolutional neural network (CNN) is a useful tool in image analysis. In order to utilize this tool, a large number of labelled data are required `train' a neural network, which would perform a specific task based on the designed of the net.

It is known that a slightly worse model would out perform a more `advanced' model given the proper data \cite{halevy2009unreasonable}. However, acquiring the large number of data required for training neural network is a daunting task. Most semiconductor images are confidential properties, and annotating the images for training purpose requires tedious effects and is proned to human error. It would be convenient to generate the images programmatically, in which case an unlimited number of images can be acquired. 

JMONSEL \cite{villarrubia2015scanning}, developed by John Villarrubia and Nicholas Ritchie of NIST, is a single--threaded electron tracker written in Java. It simulates the physical interactions between electron and the surrounding materials as it traverses through the material. Using the simulation results from JMONSEL, it is possible to construct realistic electron microscope images. However, the program runs very slowly, partly due to the fact that it is written in Java. The program needs to be modified in order to generate enough images in a feasible amount of time for training neural network.

This report details the concept, construction, verification and working details of a multi-threaded, CUDA--compliant C++ version of JMONSEL, CudaMONSEL, that is able to rapidly generate the SEM images required. Initial tests shows that CUDAMONSEL is up to $5$ times faster than JMONSEL when running on a single--thread, and the number of threads are only limited by the number hardware of the device running the simulator.