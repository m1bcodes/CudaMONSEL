\chapter{Physics}\label{sec:physics}
\section{Interaction Cross Section}
When two particles interact, their mutual cross section is the area transverse to their relative motion within which they must meet in order to scatter from each other. Interaction cross section measures the interaction rates between particles. One can imagine a beam of particles of type $a$ with flux $\phi_a$ crossing a region of space where there are $n_b$ particles per unit volume of particle $b$. The interaction rate per target particle $r_b$ will be proportional to the incident particle flux so that $r_b=\sigma\phi_a$. The proportionality constant $\sigma$, which has the dimension of area and therefore known as interaction cross section, contains the physics of the interaction.

To calculate the probability of interaction $\delta P$ when particle $a$ travelling in $v_a$ passes through a volume with cross sectional area $A$ filled with particle $b$ traveling in $v_b$ in time $\delta t$, we use
\begin{align*}
    \delta P
    &= \frac{\delta N \sigma}{A} \\
    &= \frac{n_b(v_a+v_b)A\sigma \delta t}{A} \\
    &= n_bv\sigma\delta t
\end{align*}

Thus the rate $r_a$ of interaction is given by $r_a = dP/dt=n_b v\sigma$. For a beam of particle of type $a$ with density $n_a$ confined to volume $V$, total rate is given by $r_a n_a V=\phi N_b\sigma$ (flux $\phi$ times number of target particles $n_a$ times cross section $\sigma$) and the cross section is given by number of interactions per time per target particle divided by incident flux.

In general, the cross section can be expressed as the ratio between the ``number of interest'' per unit time per target particle and the incidence flux. In this case, the number of interest is the number of interactions.

\subsection{Differential Cross Section}
When the cross section $\sigma$ depends on some final--state variable $Q$, such as angle of departure and energy at departure, it is known as differential cross section $d\sigma /dQ$. The quantity $Q$ is usually sensitive to the underlying physics, meaning it changes drastically based on the type of interaction between the source particle and the target particle. Two of the most common `Q's are final energy and angle of departure of the source particle.

Differential and total scattering cross sections are among the most important measurable quantities in nuclear, atomic, and particle physics \cite{dcs}.
