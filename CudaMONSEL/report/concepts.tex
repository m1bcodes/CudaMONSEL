\chapter{Concepts}
\label{sec:concepts}
The project's final goals is to simulate the trajectory of a electron from creation to destruction in a computational efficient manner so that large quantity of synthesized images can be obtained in feasible time to be used for training of neural networks. This section contains the background and algorithm for tracking each electron throughout its lifetime. The algorithm is being developed jointly by John S.  Villarrubia and Nicholas W. M. Ritchie of NIST.

\section{Simulation}\label{sec:concept_monte_carlo_ss}
%\begin{figure}[h]
%\begin{center}
%\includegraphics[width=0.7\textwidth]{img/Propagation_Concept.pdf}
%\caption{\label{fig:Propagation_Concept}An one-scatterer propagation environment}
%\end{center}
%\end{figure}

The simulation tracks the propagation of an electron trversing through regions. An electron is spawned by an electron gun targeted at a specific position on the material. Any statistics generated by the physical interation between the electron fired and the material is attributed to the targetd location. During the propagation, the electron may interact with the material in the region, or spawn a new electron while crossing the boundary between two regions, depending on the corresponding scattering mechanisms or barrier scattering mechanism. At each time step, the energy loss and position of the electron is updated. When a new electron is spawned, the original electron is stored on a stack until the new electron is destroyed when its energy falls below the tracking threshold of the scattering mechanism.

\section{Material}\label{sec:concept_material}
%\begin{align*}
%   S &= \sum_{t} R_r(t,r_c(t))e^{-\frac{j 2 \pi r f_c }{c}} \\
%     &= \sum_{t}\sum_{k=1}^{N(t)} A_k(t) R_s(r_c(t)-r_k(t))e^{-\frac{j 2 \pi r f_c }{c}}
%\end{align*}

\section{Material Scatter Model}\label{sec:msm}
A material scatter model consists of three parts:
\begin{enumerate}
\item A list of scattering mechanisms (e.g., Mott elastic scattering, Moller SE production, Plasmon SE production,...) that operate in the material. These scattering mechanisms may discontinuously (i.e., at a scattering event) change the primary electron energy and direction and they may create secondary electrons.
\item A single barrier scattering function. This method models scattering at a material interface.
\item A single continuous slowing down function. Determines the energy loss of the primary electron within the material. 
\end{enumerate}
Each material in the sample must be associated with a material scattering model. It combines the various mechanisms to determine overall scattering behavior, including free path, secondary generation, etc. 

\section{Scatter Mechanism}\label{sec:sm}
A scatter mechanism is a physical event which governs the motion of electron. Each material scatter model can hold multiple scatter mechanisms, each with its own scatter effects and probability of occuring (eg. scattering rate). Its effects are a function of the medium/material and the electron's current energy. Effects can include changing the (primary) electron's direction and energy, and even generating a secondary electron with its own energy and direction through some inelastic scattering events. 

Scatter mechanisms included in JMONSEL are:
\begin{enumerate}
\item BrowningMottElasticSM
\item FittedInelSM
\item GanachaudMokraniPhononInelasticSM
\item GanachaudMokraniPolaronTrapSM
\item KoteraPlasmonInelasticSM
\item MollerInelasticSM
\item SelectableElasticSM
\item TabulatedInelasticSM
\end{enumerate}
