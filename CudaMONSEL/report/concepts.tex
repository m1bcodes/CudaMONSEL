\chapter{Concepts}
\label{sec:concepts}
The project's final goals is to simulate the trajectory of a electron from creation to destruction (defined by the falling of electron energy below a threashold) in a highly computational efficient manner so that large quantity of images derived from the simulation can be obtained in feasible time and used for training of neural network. In order to do this, the code should be as efficient as possible and any computation resource available should fully--utilized. This section contains the background and algorithm for tracking each electron through its lifetime. The algorithm was developed--jointly by John S.  Villarrubia and Nicholas W. M. Ritchie using a variety of techniques from particle physics.

\section{Simulation}\label{sec:concept_monte_carlo_ss}
%\begin{figure}[h]
%\begin{center}
%\includegraphics[width=0.7\textwidth]{img/Propagation_Concept.pdf}
%\caption{\label{fig:Propagation_Concept}An one-scatterer propagation environment}
%\end{center}
%\end{figure}

The simulation tracks the propagation of an electron trversing through regions. An electron is spawned by an electron gun targeted at a specific position on the material. Any statistics generated by the physical interation between the electron fired and the material is attributed to the targetd location. During the propagation, the electron may interact with the material in the region, or spawn a new electron while crossing the boundary between two regions, depending on the corresponding scattering mechanisms or barrier scattering mechanism. At each time step, the energy loss and position of the electron is updated. When a new electron is spawned, the original electron is stored on a stack until the new electron is destroyed when its energy falls below the tracking threshold of the scattering mechanism.

\section{Material}\label{sec:concept_material}
%\begin{align*}
%   S &= \sum_{t} R_r(t,r_c(t))e^{-\frac{j 2 \pi r f_c }{c}} \\
%     &= \sum_{t}\sum_{k=1}^{N(t)} A_k(t) R_s(r_c(t)-r_k(t))e^{-\frac{j 2 \pi r f_c }{c}}
%\end{align*}

